%%% Title:    Introduction to R Winter School: Introductory Lecture
%%% Author:   Kyle M. Lang
%%% Created:  2016-01-28
%%% Modified: 2023-01-23

\documentclass[10pt]{beamer}
\usetheme{Utrecht}

\usepackage{graphicx}
\usepackage[natbibapa]{apacite}
\usepackage[libertine]{newtxmath}
\usepackage{booktabs}
\usepackage{caption}
\usepackage{url}

\newcommand{\rmsc}[1]{\textrm{\textsc{#1}}}
\newcommand{\pkg}[1]{\textbf{#1}}
\newcommand{\code}[1]{\texttt{#1}}

\title{Introduction}
\subtitle{Utrecht University Winter School: Introduction to R}
\author{Kyle M. Lang}
\institute{Department of Methodology \& Statistics\\Utrecht University}
\date{}

%------------------------------------------------------------------------------%

\begin{document}

\begin{frame}[t, plain]
  \titlepage
\end{frame}

%------------------------------------------------------------------------------%
\comment{%%%%%%%%%%%%%%%%%%%%%%%%%%%%%%%%%%%%%%%%%%%%%%%%%%%%%%%%%%%%%%%%%%%%%%
\begin{frame}{Outline}
  \tableofcontents
\end{frame}

%------------------------------------------------------------------------------%

\section{Introduction Round}

%------------------------------------------------------------------------------%

\begin{frame}{Introduction Round}
  
  \begin{enumerate}
  \item Name
  \item Affiliation/Home base/Type of work
  \item Statistical background
  \item Programming background
  \item (Currently) preferred statistical analysis software
  \item Why do you want to learn R?
  \item Favorite hobby
  \end{enumerate}
  
\end{frame}
}%%%%%%%%%%%%%%%%%%%%%%%%%%%%%%%%%%%%%%%%%%%%%%%%%%%%%%%%%%%%%%%%%%%%%%%%%%%%%%%

%------------------------------------------------------------------------------%

\section{Plan for the Day}

%------------------------------------------------------------------------------%

\begin{frame}{Plan for the Day}

  After this lecture, we'll spend the rest of the day working through
  interactive R scripts.
  \begin{itemize}
  \item The scripts live in the ``code'' directory.
  \item You should follow along.
  \item You will solve embedded practice problems.
  \end{itemize}
  \vb
  We'll stop for a one-hour lunch break between 13:00 and 14:00.
  \begin{itemize}
  \item We'll take a few other breaks in between.
  \end{itemize}
  \vb
  The course will finish at 18:00.
  \begin{itemize}
  \item You're free to access the course materials for as long as you like.
  \end{itemize}
  
\end{frame}

%------------------------------------------------------------------------------%
\comment{%%%%%%%%%%%%%%%%%%%%%%%%%%%%%%%%%%%%%%%%%%%%%%%%%%%%%%%%%%%%%%%%%%%%%%%  
\sectionslide{Open-Source Software}

%------------------------------------------------------------------------------%

\begin{frame}{What is ``Open-Source''?}

  R is an open-source software project, but what does that mean?
  \va
  \begin{itemize}
  \item Source code is freely available to anyone who wants it.
    \vb
    \begin{itemize}
    \item Free Speech, not necessarily Free Beer
    \end{itemize}
    \vb
  \item Anyone can edit the original source code to suit their needs.
    \vb
    \begin{itemize}
    \item Ego-less programming
    \end{itemize}
    \vb
  \item Many open source programs are also ``freeware'' that are available free
    of charge.
    \vb
    \begin{itemize}
    \item R is both open-source and freeware
    \end{itemize}
  \end{itemize}

\end{frame}

%------------------------------------------------------------------------------%

\begin{frame}{Strengths of Open-Source Software}

  \rmsc{Freedom}
  \vb
  \begin{itemize}
  \item If the software you are using is broken (or just limited in capability), 
    you can modify it in any way you like.
    \vb
  \item If you are unsure of what the software you are using is doing, you can 
    dig into the source code and confirm its procedures.
    \vb
  \item If you create some software, you can easily, and independently, 
    distribute it to the world.
    \vb
    \begin{itemize}
    \item There is a global community of potential users that are all linked via 
      a common infrastructure that facilitates open-source software development 
      and distribution.
    \end{itemize}
  \end{itemize}

\end{frame}

%------------------------------------------------------------------------------%

\begin{frame}{Strengths of Open-Source Software}

  \rmsc{Peer Review}
  \vb
  \begin{itemize}
  \item Every user of open-source software is a reviewer of that software.
    \vb
  \item What ``bedroom programmers'' lack in term of quality control procedures 
    is overcome by the scrutiny of a large and empowered user-base.
    \vc
    \begin{itemize}
    \item When we use closed source software, we are forced to trust the honesty 
      of the developing company.
      \vb
    \item We have no way of checking the actual implementation.
    \end{itemize}
  \end{itemize}

\end{frame}

%------------------------------------------------------------------------------%

\begin{frame}{Strengths of Open-Source Software}

  \rmsc{Accessibility}
  \vb
  \begin{itemize}
  \item Many open-source programs (like R) can be downloaded, for free, from the 
    internet.
    \begin{itemize}
      \vb
    \item You can have R installed on all of you computers (and your mobile 
      phone, your car's info-tainment system, your microwave, your clock-radio,
      ...).
      \vb
    \item No need to beg, borrow, or steal funds to get yourself up-and-running 
      with a cutting-edge data analysis suite.
    \end{itemize}
    \vb
  \item Licensing legality is very simple---no worries about being sued for 
    installing open-source software on ``too many'' computers.
    \vb
  \item Open-source software tends to run on more platforms than closed-source 
    software will.
  \end{itemize}

\end{frame}

%------------------------------------------------------------------------------%

\subsection{Open-Source Licensing}

%------------------------------------------------------------------------------%

\begin{frame}{A Note on Licensing}

  Some popular open-source licenses:
  \vb
  \begin{itemize}
  \item The GNU General Public License (GPL)
    \begin{itemize}
    \item \url{http://www.gnu.org/licenses/gpl-3.0.en.html}
    \end{itemize}
    \vc
  \item The GNU Lesser General Public License (L-GPL)
    \begin{itemize}
    \item \url{http://www.gnu.org/licenses/lgpl-3.0.en.html}
    \end{itemize}
    \vc
  \item The Apache License
    \begin{itemize}
    \item \url{http://www.apache.org/licenses/}
    \end{itemize}
    \vc
  \item The BSD 2-Clause License (FreeBSD License)
    \begin{itemize}
    \item \url{http://opensource.org/licenses/BSD-2-Clause}
    \end{itemize}
    \vc
  \item The MIT License
    \begin{itemize}
    \item \url{https://opensource.org/licenses/MIT}
    \end{itemize}
  \end{itemize}

\end{frame}

%------------------------------------------------------------------------------%

\begin{frame}{A Note on Licensing}

  Many open-source licenses (e.g., GPL, L-GPL) ``copyleft'' their products.
  \vb
  \begin{itemize}
  \item Copyleft is designed to ensure that open-source software cannot be 
    closed.
    \vc
    \begin{itemize}
    \item I can't take your copylefted software, repackage it, and sell it in 
      violation of your original licensing terms.
    \end{itemize}
  \end{itemize}
  \va
  Other open-source licenses (e.g., BSD-Types, Apache, MIT) are non-copyleft, 
  ``permissive'' licenses.
  \vb
  \begin{itemize}
  \item Many of these licenses are designed to promote commercialization of 
    open-source products.
    \vc
    \begin{itemize}
    \item E.g., allowing a student to develop a company selling a product they 
      developed for their dissertation
    \end{itemize}
  \end{itemize}
  
\end{frame}

%------------------------------------------------------------------------------%

\sectionslide{The R Statistical Programming Language}
}%%%%%%%%%%%%%%%%%%%%%%%%%%%%%%%%%%%%%%%%%%%%%%%%%%%%%%%%%%%%%%%%%%%%%%%%%%%%%%%
%------------------------------------------------------------------------------%

\section{What is R?}

%------------------------------------------------------------------------------%

\begin{frame}{What is R?}
  
  R is a holistic (open-source) software system for data analysis and
  statistical programming.
  \vc
  \begin{itemize}
  \item R is an implementation of the S language.
    \begin{itemize}
    \item Developed by John Chambers and colleagues 
      \begin{itemize}
      \item \citet{beckerChambers:1984}
      \item \citet{beckerEtAl:1988}
      \item \citet{chambersHastie:1992}
      \item \citet{chambers:1998}
      \end{itemize}
    \end{itemize}
    \vc
  \item Introduced by \citet{ihakaGentleman:1996}.
    \begin{itemize}
    \item Currently maintained by the \emph{R Core Team}.
    \end{itemize}
    \vc
  \item Support by thousands of world-wide contributors.  
    \begin{itemize}
    \item Anyone can contribute an R package to the \emph{Comprehensive R 
        Archive Network} (CRAN) 
    \item Must conform to the licensing and packaging requirements.
    \end{itemize}
  \end{itemize}
  
\end{frame}

%------------------------------------------------------------------------------%

\begin{frame}{What is R?}

  I prefer to think about R as a \emph{statistical programming language}, rather 
  than as a data analysis program.
  \vb
  \begin{itemize}
  \item R \textbf{IS NOT} its GUI (no matter which GUI you use).
    \vb
  \item You can write R code in whatever program you like (e.g., RStudio, EMACS, 
    VIM, Notepad, directly in the console/shell/command line).
    \vb
  \item R can be used for basic (or advanced) data analysis, but its real 
    strength is its flexible programming framework.
    \vc
    \begin{itemize}
      \item Tedious tasks can be automated.
        \vc
      \item Computationally demanding jobs can be run in parallel.
        \vc
      \item R-based research \emph{wants} to be reproducible.
        \vc
      \item Analyses are automatically documented via their scripts.
    \end{itemize}
  \end{itemize}

\end{frame}

%------------------------------------------------------------------------------%

\begin{frame}{Getting R}

  You can download R, for free, from the following web page:
  \va
  \begin{itemize}
  \item \url{https://www.r-project.org/}
  \end{itemize}
  \va
  You will also need a proper text editor/IDE. For those who are just learning R, 
  I recommend \pkg{RStudio}:
  \va
  \begin{itemize}
  \item \url{https://www.rstudio.com/}
  \end{itemize}

\end{frame}

%------------------------------------------------------------------------------%

\section{Using R}

%------------------------------------------------------------------------------%
\comment{%%%%%%%%%%%%%%%%%%%%%%%%%%%%%%%%%%%%%%%%%%%%%%%%%%%%%%%%%%%%%%%%%%%%%%%
\begin{frame}{What to Expect when Opening R}

  First things first, let's take a look at a few ways you can interact with R:
  \vb
  \begin{itemize}
  \item Base R
    \vb
  \item EMACS
    \vb
  \item RStudio
    \vb
  \item Text-only console (i.e., even more base R)
  \end{itemize}

\end{frame}
}%%%%%%%%%%%%%%%%%%%%%%%%%%%%%%%%%%%%%%%%%%%%%%%%%%%%%%%%%%%%%%%%%%%%%%%%%%%%%%%
%------------------------------------------------------------------------------%

\begin{frame}{How R Works}

  R is an interpreted programming language.
  \vb
  \begin{itemize}
  \item The commands you enter into the R \emph{Console} are executed 
    immediately.
    \vc
  \item You don't need to compile your code before running it.
    \vc
  \item In this sense, interacting with R is similar to interacting with other
    syntax-based statistical packages (e.g., SAS, STATA, Mplus).
  \end{itemize}

\end{frame}

%------------------------------------------------------------------------------%

\begin{frame}{How R Works}
  
  R mixes the \emph{functional} and \emph{object-oriented} programming
  paradigms.
  \vc
  \begin{columns}
    \begin{column}{0.5\textwidth}
      
      \begin{center}
        \rmsc{Functional}
      \end{center}
      
      \begin{itemize}
      \item R is designed to break down problems into functions.
        \vc
      \item Every R function is a first-class object.
        \vc
      \item R uses pass-by-value semantics.
      \end{itemize}
      
    \end{column}
    \begin{column}{0.5\textwidth}
      
      \begin{center}
        \rmsc{Object-Oriented}
      \end{center}
      
      \begin{itemize}
      \item Everything in R is an object.
        \vc
      \item R functions work by creating and modifying R objects.
        \vc
      \item The R workflow is organized by assigning objects to names.
      \end{itemize}
      
    \end{column}
  \end{columns}
  
\end{frame}

%------------------------------------------------------------------------------%

\begin{frame}{Interacting with R}

  When working with R, you will write \emph{scripts} that contain all of the 
  commands you want to execute.
  \va
  \begin{itemize}
    \item There is no ``clicky-box'' Tom-foolery in R.
    \vb
    \item Your script can be run interactively or in ``batch-mode'', as a 
      self-contained program.
  \end{itemize}
  \va
  The primary purpose of the commands in your script will be to create and 
  modify various objects (e.g., datasets, variables, function calls, graphical 
  devices).

\end{frame}

%------------------------------------------------------------------------------%

\begin{frame}{What's Next?}

  Now, we'll start in on the meat-and-potatoes of the course
  \begin{itemize}
  \item The interactive R scripts
  \end{itemize}
  \vb
  We have seven potential topics:
  \begin{enumerate}
  \item Basic commands
  \item Data objects
  \item Data I/O
  \item Data manipulation
  \item Data analysis
  \item Data visualization
  \item Simple programming
  \end{enumerate}
  
\end{frame}

%------------------------------------------------------------------------------%

\begin{frame}[allowframebreaks]{References}
  \bibliographystyle{apacite}
  \bibliography{../../bibtex/winter_school_refs.bib}
\end{frame}

%------------------------------------------------------------------------------%

\end{document}
